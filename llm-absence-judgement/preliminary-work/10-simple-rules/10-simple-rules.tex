Iterate over prompts with an appropriate model...

Check all of your references

Remember that design of prompts and interpretation of results depends on your expertise

Don't give it too much at once, decompose tasks. 

Seemingly minor changes to prompts can have significant impact (better or worse) on outputs

Iterate over prompts until you get the results you are looking for.

Use an LLM to help you write prompts (meta-prompting).

Use multiple models. Models change frequently and each has its own personality. Each may be good for a different set of tasks. Compare competing tools (e.g., the 'research' product offered by each of the major labs). Test the same prompts across different models.

Use APIs and build tools. 'Mundane' software scaffolding often improves outputs, allowing modifications of settings unavailable in the chatbots (enumerate). Huggingface at GitHub have many useful workflows / pipelines.

Spend at least one hour per week reading AI blogs / substacks / listening to podcasts / etc. 

Use system settings. You can tailor your language preferences, adversariality of the model, etc. But know when to use Project information vs. global settings vs. chat-specific instructions.

Don't give orders that won't be followed (e.g., Claude Research CSV generation)

Develop a Writing Style Guide and a Research Philosophy Guide

Aim for / expect 'better than nothing'

Always check settings before submitting the prompt. Do you need 'web search', 'thinking', and/or extended 'research'? Are you sure?

Go with the flow - if repreated prompt modfifications don't change a behaviour, consider incorporating that behaviour if possible (example: Claude wouldn't stop talking about old versions of software no matter how we tried to stop it, so we added a 'history' section and specified that talking about the old versions is fine, so long as current version is first and each observation is specified to a particular version)

Ask for a plan for a thing (writing, code, etc.) before doing the thing.

Metaprompting...ask the model for help with prompts

As with other automation, time spent on prompts should reflect time spent doing the thing yourself (if it takes longer to write the prompt, just do the thing).

Be wary of 'pollution' / cross-contamination / bleed-through - don't suggest answers, don't transfer content with style instructions, and watch for it leaking through elsewhere. etc.

Use lots of prompts with lots of models to get a feel for capability and personality.

...add more as we think of them, hopefully we'll get more than 10 and can pick the best ones