Hi DR, Today we're going to be doing a deep and systematic SEARCH. This is one component of a much longer process, so you're going to need to be specific, focused, meticulous, and detail-oriented.

Please look at open-archaeo https://open-archaeo.info/. Note that 'open-archaeo' is a listing of digital resources for archaeology. Any item may (or may not) be a software 'tool'. The name of the potential tool is the title of the item in open-archaeo. Don't search the text of the items for tools, consider the title of each item a possible tool. 

Your job is to:

(1) CREATE A LIST of ALL items in Open-Archaeo that are in the following categories.

- PACKAGES AND LIBRARIES (236 items)
- PRODUCTS (15 items)
- SCRIPTS (67 items)
- STANDALONE SOFTWARE (73 items)

(2) FLAG items meet our definition of a 'tool'. Some items are only websites, instructions, guides, or other resources that are not tools, and you should screen these out. 

(3) IMPROVE the description each item by finding additional informaton online (when available) and explaining what the tool does.

Items tagged as 'Packages and libraries', 'Scripts', and 'Standalone software' are PROBABLY but not certainly tools by the definition included below, but still need to be checked.  

'Products' may or may not be tools by our definition, so you will need to evaluate on a case-by-case basis with more scrutiny. 

Note that all items are only listed under one category.

# Tool definition

For our project, a software tool is any discrete piece of software—be it a program, script, or web application—that is developed or substantially modified by researchers to perform specific, research-oriented computational tasks. These tasks must involve active data processing, such as analysis, simulation, visualization, modeling, or automation, where input data is dynamically created, documented, transformed, or interpreted to generate meaningful results. In contrast, products whose primary function is to simply display, store, or promote static datasets—without offering mechanisms for active computational engagement—are excluded. 

In essence, a valid software tool for our purposes is one that not only **supports research in archaeology or historical sciences** but does so by actively collecting and/or processing data rather than merely serving as a generic data repository or static interface.

Use this definition to evaluate whether or not an open-archaeo item is a 'tool'. 

**Data to Extract for Each item:**  

- **Title:** The item name, enclosed in quotes.  
- **Description:** The item description from Open-Archaeo
- **Category:** The category the tool is tagged with in Open-Archaeo
- **Authors:** The article’s authors, listed in a dedicated column.  
- **Open-Archaeo URL:** The URL of the open-archaeo page.
- **Improved description:** An expanded description incorporating the Open-Archaeo description plus anything else you can find about the tool online. If no additional information is available online, flag that here with 'no additional information'.
- **Tool:** A yes/no field flagging whether or not an item is a tool, based on the 'Improved description'.
- **AI notes:** an explanation of your decision plus any notes, concerns, or uncertainties you have. 

**Output Format:**  
The final dataset should be output as a CSV file INSIDE A FENCED CODE BLOCK. The columns must appear in this exact order:

"Title" "Category" "Authors" "Open-Archaeo URL" "Tool (y/n)" "AI Notes"


**Additional Instructions:**  

- **No inline citations:** Avoid any extra markers or citation references within the data cells to prevent data corruption.  
- **Meticulous research:** Take as much time as needed to carefully review each item and verify your accuracy.  
- **Error Reporting:** If any errors or issues arise during the process, provide a separate error log outlining the encountered problems.  
- **Iterative Process:** This is a long-term, iterative project, so precision is prioritized over speed.  

Finally, after the CSV I'd like any additional comments or observations. Please include a header 'Comments', then provide observations about your results. 

Pleae confirm you understand, and ask any questions you need to clarify these instructions.