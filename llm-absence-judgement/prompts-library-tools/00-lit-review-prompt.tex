Hi [model], I would like you to be my PhD-level digital research assistant and help with the following literature review.

Objective:

Produce a comprehensive and scholarly literature review on the longevity, lifecycle, and sustainability of research software in historical studies. This review should cover digital humanities, archaeology, history, historical linguistics, corpus linguistics, and textual studies. When relevant, also incorporate insights from research software sustainability in other academic domains (e.g., physical sciences, computer science, engineering) and, where necessary, summarize key points from literature about commercial software - but only where academic research is lacking or for contextual comparison. Emphasize two main aspects:

Empirical studies that shed light on the lifespan of research software (including those from non-historical domains and non-academic contexts if there is a shortage of examples from historical studies), and End-of-life planning and mitigation strategies that inform software design practices to support long-term data interoperability and minimally disrupted user workflows after software reaches its end-of-life.

Note: This literature review will inform a paper that (a) empirically studies software lifespan in historical studies, and (b) advocates for designing research software with an inherent awareness of its eventual obsolescence.

Scope and Structure:

Overview of the Field:

-Summarize the current state of research on the lifecycle and sustainability of research software in historical studies.
-Describe the key challenges (technical, institutional, financial, human) that affect software longevity.
-Include a brief discussion of insights from other domains and how challenges in historical studies compare to those in other domains.

Existing Frameworks and Methodologies:

-Detail existing frameworks and models (e.g., maturity models, FAIR principles for software, national sustainability reports) relevant to research software.
-Discuss case studies from digital humanities and related fields, including examples such as the “Research Software Levels” framework and initiatives like the UK Software Sustainability Institute and Software Heritage archives.
-Highlight how these frameworks have been applied (or not) within historical studies.

Empirical Evidence and Case Studies:

-Analyze both quantitative and qualitative evidence regarding the lifespan of research software.
-Include examples from studies that have mined code repositories (e.g., GitHub, Zenodo) or used archival tools (e.g., Internet Archive) to assess software longevity.
-Critically evaluate the methodologies used in these empirical studies, noting any limitations or biases.

Comparative Analysis:

-Compare the research software sustainability issues observed in historical studies with findings from other fields.
-Discuss methodological differences and highlight best practices from domains that have successfully addressed long-term maintenance and preservation.
-Include a discussion of how insights from commercial software (when applicable) can inform research software sustainability, keeping the focus primarily on academic contexts.

Identification of Gaps and Future Directions:

-Identify key gaps in the current literature, such as the absence of a unified methodology for tracking software lifecycles, lack of empirical studies (to the extent such a lack exists), or a standardized metric for “software lifespan.”
-Discuss the potential reasons behind these gaps and suggest specific directions for future research.
-Include a section on the implications of these gaps for both empirical research and software design practices.

Methodology and Source Verification:

-Use only high-quality sources: peer-reviewed journals, books from reputable publishers, credible white papers, and grey literature with recognized authority.
-Examine sources systematically - be as comprehensive as possible within the quality limits established above.
-Explain the criteria used for selecting literature (e.g., peer-reviewed status, impact factor, citation count) and any search strategies employed (including search terms and inclusion/exclusion criteria).
-Utilize academic databases (e.g., Google Scholar, JSTOR), inspect relevant source code repositories (e.g., GitHub, Zenodo), consult archival records (e.g., Internet Archive), and employ persistent identifiers (e.g., Crossref, DataCite) to verify and supplement references.

Documentation and Referencing:

-Provide detailed and accurate citations in an appropriate academic style (e.g., Oxford or Chicago) for all references.
-Include annotations, summaries, and keywords for each major reference to explain its relevance and context within the review.
-In addition to a human-readable bibliography, produce a BibTeX reference list (suitable for import into Zotero) that includes annotations (as notes) and keywords (as tags). Make these references as complete as possible, including all authors (not just 'et al' or 'and others', source DOIs and URLs, and abstracts where available.

Output Requirements:

-The final literature review should be structured with clear headings and subheadings that reflect the sections outlined above (unless you determine a superior organisational approach arising from your research).
-Write in a moderately formal academic style with detailed explanations, ensuring the content is suitable for publication in a top-quartile journal.
-Present and discuss multiple perspectives on key issues, providing a balanced analysis of the evidence.
-Conclude with a synthesis of the findings and a discussion of future research directions, emphasizing actionable insights for both empirical study and software design.

Tools and Capabilities:

-You have full access to live academic databases, scholarly websites, persistent identifiers, and source code repositories, as well as archival services like the Internet Archive.
-Use these capabilities to ensure that all referenced material is current, accurate, and thoroughly vetted.
-Highlight any significant debates and historical or emerging trends discovered during the review process.

Final Deliverable:

A comprehensive, well-structured literature review that meets all the requirements described above, ready for inclusion in a scholarly publication.

Clarifications:

1. Include literature from the mid-1990s to the present, with emphasis on the last 20 years (2005-2025).
2. See what you find, but include major multidisciplinary journals (e.g., PLOS, Science, Nature, PNAS, Royal Society journals), plus digital humanities, linguistics, and archaeological journals (e.g., Literary and Linguistic Computing, Digital Humanities Quarterly, Digital Scholarship in the Humanities, Internet Archaeology, Antiquity, Journal of Archaeological Science, Journal of Field Archaeology, and the journal and conferences of Computer Applications and Quantitative Methods in Archaeology (CAA) ). Please do seek out other relevant journals, however, as this list is not comprehensive. Use sources like Scimago and Web of Science to find journals the cover the appropriate domains.
3. Please search for appropriate and instructive case studies.
4. I am interested in both emperical and conceptual aspects of the question. It is important to cover empirical studies comprehensively, however, as this lit review is going to support and contextualise an empirical study.
5. Please also include research in other languages; I can read enough French and German to check sources in those languages (if you find highly cited literature in other languages you can include that).
6. No length limits at this point; I would prefer a longer review that we can work together to pare down if needed.

Thank you!