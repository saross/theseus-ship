Hi DR. Today we are going to be collecting evidence for a research paper. 

# Ground rules

ABOVE ALL ELSE, if you are unsure about something, or do not have evidence from a web page supporting a claim, do not make a claim. If there is any doubt, there is no doubt.

This is research. SPEND EXTRA TIME above and beyond normal, to collect data carefully and methodically. We don't care about how long it takes, only that your research is accurate and epistemologically grounded. Care and attention to detail second only to the prior rule.

# Definitions

## Tool definition
For our project, a software tool is any discrete piece of software—be it a program, script, or web application—that is developed or substantially modified by researchers to perform specific, research-oriented computational tasks. These tasks must involve active data processing, such as analysis, simulation, visualization, modeling, or automation, where input data is dynamically created, documented, transformed, or interpreted to generate meaningful results. In contrast, products whose primary function is to simply display, store, or promote static datasets—without offering mechanisms for active computational engagement—are excluded. 

In essence, a valid software tool for our purposes is one that not only **supports research in archaeology or historical sciences** but does so by actively collecting and/or processing data rather than merely serving as a generic data repository or static interface.

It is our belief that the tools we have given you meet this criteria, however, if there is an absence of evidence supporting our belief, it is REQUIRED that you inform us of any problems. IF YOU ARE UNSURE, INDICATE IT rather than giving an answer.

# Task

I will be giving you the name or acronym of an archaeology and historical studies-related software tool which dates somewhere from 1995 onwards. 

You must build a well-formed quoted CSV inside a fenced code block with the following columns:
csv
"Tool", "Year", "Source", "URL", "AI Notes"

* Tool is our key column. It is essential to be consistent in tool name, following the input we give you. If there is an issue indicate it in notes.
* Year, year from the web page, article, or source you found, n.d. is a valid entry for "no date"
* Source: Citation or (Domain + Page Title)
* URL, url of the resource. 
* AI notes: any notes, concerns, or uncertainties you have with the data provenance or quality. Do not use HTML citations.

This is a comprehensive and systematic search. For each year, 1995 onwards, we one mention (one line) per source found. Obviously, many of these tools were created after 1995, so we'll want to expand our search by +- 1 year from any date we find. When you find a source, systematically explore adjacent years in your searches.

DO NOT extrapolate dates from sources. ONLY include dates where a source has been published in that year. We will perform all necessary inference, if you include a line for EACH valid source you find. For undated items, indicate "n.d."

This research requires that you be thoughtful, systematic, and take your time. You will likely need to visit EVERY SINGLE PAGE of search results for a given tool. 

To repeat back.

I'm going to give you a list of tools, one by one.
You will search for them.
For every source you find,  we want to record tool, year/n.d., source (citation or title), and url

You MUST build a CSV with the output in your direct response to the deep research process.

Allowed sources: 
* Do try to find it's source code in published version control repos (add a row if they have public code, as the commits will have dates)
* do try to find scholarly literature (each to one row)
* do include blog posts and about pages (each to one row)
* do include any other mentions of the tool (each to one row)

When you respond with data:
RESULTS: Build a markdown **fenced code block** CSV of evidence you have found with  ANY mention of that tool in ANY year/n.d. should be on the CSV, regardless of source. Include one line for EACH source you find. Do not include any inline citations automatically generated by the deep research process (i.e., AI-generated or system-injected references i.e. 【deep research】 or &#8203;:contentReference[oaicite). 

Do you understand?
