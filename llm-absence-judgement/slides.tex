
Set stage with the Theseus' Ship scenario
In about 2015(!) my FAIMS colleagues and I started a paper on software longevity in archaeology, looking at the typical lifespan of tools and arguing that projects building research software need to have end-of-life plans from the beginning and emphasise interoperability so that any data they are collecting / managing / processing can be 'passed on' to other tools. This paper repeatedly stalled, mostly because we needed evidence - a dataset of archaeological software tools that captured their lifespan - and building one was very slow. We needed an RA to work on it, but could never get one.

In February we decided to try it (done with Claude 3.7 / ChatGPT 4.5 / o3 / Gemini 2.5 generation models), and at the same time assess whether or not LLMs are ready for prime time when it comes to serious research, and if so, across what research lifecycle phases.


Bibliography \& lit review
Dataset - decompose the task. Run it slowly and deliberately.
Engineering the dataset prompts
Idea development - Brian's workflow
Writing - Shawn's writing style guide


Different models are applicable at different stages of the process
Agents are LLMs calling tools in a loop

Do prompts and outputs with a buddy - if team wants to get better, do it together


Mark and Brian Asian Journal of Philosophy Proleptic Reasoning
Toulmin analysis
Ask one question at a time - when you want to make progress on an idea

AI locus of control
-using it a tool instead of a cheating engine
-augmentation vs. automation
-'three types of people in the world'